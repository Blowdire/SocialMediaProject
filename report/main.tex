% Social Media Analytics 2022/23

%%%%%%%%%%%%%%%%%%%%%%%%%%%%%%%%%%%%%%%%%%%%%%%%%%%%%%

% OPTIONAL PACKAGES
\documentclass[12pt,journal,compsoc]{IEEEtran}
\usepackage{amsfonts}
\usepackage{caption}
\usepackage{multirow}
\usepackage[table,xcdraw]{xcolor}
\usepackage{booktabs}
\usepackage{tabularx}
\usepackage[hyphens]{url}
\usepackage{biblatex}
\usepackage{blindtext,graphicx}
\usepackage[absolute]{textpos}
\usepackage[italian]{babel}

\addbibresource{ref.bib} %Import the bibliography file

%%%%%%%%%%%%%%%%%%%%%%%%%%%%%%%%%%%%%%%%%%%%%%%%%%%%%%

\begin{document}

\begin{textblock}{5}(1,0.5)
\noindent\small Social Media Analytics - UniMib 2022/23
\end{textblock}

\title{Smart Working\\
\vspace{2mm}\large{Social Network and Content Analysis}}
\author{Agazzi Ruben 844736\\Cominetti Fabrizio 882737}

\IEEEtitleabstractindextext{%
\begin{abstract}
Negli ultimi anni il tema legato allo \textit{smart working} ha assunto una caratura di primo rilievo nelle discussioni in ambito lavorativo. Dalla necessità nel periodo di pandemia alle richieste di lavoratori che, una volta ristabilità la situazione sanitaria post-pandemia, non hanno voluto rinunciare a tale possibilità, considerata un vantaggio nel proprio equilibrio tra vita privata e lavoro. Ma il tema è discusso anche per i vantaggi offerti alle aziende, soprattutto in termini di risparmio su strutture ed energie, oppure ancora per questioni legate a traffico e inquinamento. Insomma, lo smart working non può più essere considerato un tema di secondo piano. Per questo motivo, abbiamo voluto analizzare il sentiment e le community degli utenti relativamente al tema, specialmente dopo gli ultimi interventi a livello governativo che hanno parzialmente ridotto la possibilità di usufruire dello smart working.
\end{abstract}}

\maketitle
\IEEEpeerreviewmaketitle
\IEEEdisplaynontitleabstractindextext

\tableofcontents

%%%%%%%%%%%%%%%%%%%%%%%%%%%%%%%%%%%%%%%%%%%%%%%%%%%%%%

\section{Introduction}
\IEEEPARstart{L}{o} Smart Working, anche conosciuto come lavoro agile, indica \textit{una modalità di esecuzione del rapporto di lavoro subordinato caratterizzato dall'assenza di vincoli orari o spaziali e un'organizzazione per fasi, cicli e obiettivi, stabilita mediante accordo tra dipendente e datore di lavoro; una modalità che aiuta il lavoratore a conciliare i tempi di vita e lavoro e, al contempo, favorire la crescita della sua produttività} \cite{Gazzetta} \cite{MIUR}.\\
Lo smart working è un modello organizzativo in grado di portare notevoli vantaggi alle organizzazioni che lo adottano: in termini di produttività, di raggiungimento degli obiettivi, ma anche in termini di welfare e qualità della vita del lavoratore.\\
Già nel 2015, il British Standards Institution affermava che: \textit{"I principi dello smart working riconoscono che la tecnologia e i modelli di lavoro flessibile stanno cambiando in meglio il modo in cui lavoriamo} \cite{BSI}.\\
Guardando all'altra faccia della medaglia, ricerche come quella condotta da NordVPN nel 2020, hanno portato alla luce alcuni rischi per i lavoratori, tra cui un involontario aumento di ore lavorative, l'isolamento dell'individuo e la difficoltà dello stesso di separare vita sociale e attività lavorativa, oltre a pericoli legati al tema privacy e sicurezza \cite{Forbes}.\\
Tra fine 2021 e inizio 2022, si stimavano intorno ai 2,9 milioni i cosiddetti "smart worker" in Italia, in aumento rispetto ai 1,15 milioni di fine 2019 ma in calo al giorno d'oggi e su un totale stimato di 8 milioni di potenziali usufruitori. La stessa ricerca mostra inoltre come l'Italia sia fanalino di coda nell'adozione dello smart working in Europa e abbia una media di gran lunga inferiore rispetto alla media UE considerando numero di individui e totale di giorni a settimana di lavoro agile \cite{Rai}.\\
A partire da febbraio 2020, a seguito del diffondersi dell'epidemia Covid-19 del Coronavirus, sono stati emanati dal Governo una serie di provvedimenti per semplificare l'accesso allo Smart Working e diffonderne al massimo l'utilizzo. La pandemia ha infatti costretto numerosi lavoratori a fronteggiare l'emergenza sanitaria tramite lo smart working.\\
I numeri enunciati sopra fanno pensare che, una volta finita la fase più dura dell'emergenza - con misure tra le quali il lockdown - la maggior parte di aziende e lavoratori abbiano scelto di tornare alle modalità di lavoro tradizionali.\\
L'opinione pubblica è divisa tra chi ha preferito continuare ad usufruire di una modalità di lavoro agile anche in seguito, sottolineandone i numerosi benefici, e chi, con un parere opposto, sostiene l'importanza della comunicazione face-to-face e considera insostituibile l'interazione umana giornaliera, oppure ancora vorrebbe cedere il passo a nuovi strumenti e tecnologie.\\
In aggiunta a tutto ciò, è anche opportuno precisare che solo alcune tipologie di lavoratori possono svolgere con continuità una modalità di lavoro agile, principalmente i lavoratori d'ufficio, mentre altri hanno dunque un interesse marginale per l'argomento.\\
Oltre alla già citata pandemia, le direttive annesse per i lavoratori e le varie proroghe allo stato d'emergenza e allo smart working in regime semplificato, hanno portato alla ribalta, una volta di più, il fenomeno.\\
Il Governo ha infatti prorogato negli ultimi giorni del 2022 il diritto allo smart working con un emendamento dedicato in Manovra soltanto per i lavoratori considerati "fragili" del pubblico e del privato. Dunque, dicendo addio ai giorni di lavoro da casa per i genitori di figli under 14, fino a prima di questo emendamento compresi nella misura varata dal precedente Esecutivo, e che dall’1 gennaio sono dovuti tornare a contrattare individualmente i giorni di lavoro da remoto con la propria azienda.\\
La decisione del Governo di escludere dalla misura i genitori di figli under 14 è spiegata dal venire meno delle misure anti-Covid adottate durante la pandemia, sulle quali si basava la necessità di ricorrere al lavoro da remoto per milioni di lavoratori.\\
La nostra analisi vuole dunque concentrarsi su questo periodo finale, tra la fine del 2022 e l'inizio del 2023, per indagare le reazioni, il sentiment e le emozioni riversate sulla rete - Twitter nel nostro caso - in seguito all'annuncio di queste decisioni.\\
Per concludere, il progetto vuole rispondere a domande quali \textit{Come è rappresentato il panorama italiano intorno al tema su Twitter? Quali sono gli utenti più influenti al suo interno? Qual è stato il sentiment generale nel periodo preso in considerazione? Quali sono state le emozioni e le diverse reazioni nei giorni intorno all'annuncio del Governo sull'ulteriore riduzione dello Smart Working?}

\section{Data Collection}
Per costruire il dataset, necessario per rispondere alle domande di progetto, abbiamo scaricato un totale di x tweet dal social network Twitter, tramite l'utilizzo dell'API fornita dallo stesso social.\\
I dati sono stati scaricati utilizzando un jupyter notebook e la libreria Tweepy, che permette di effettuare il download di tweet corrispondenti ad uno o più hashtag o keyword. Nel nostro caso, abbiamo scelto di utilizzare le seguenti keyword: \textit{x, x, x}.\\
Siccome il tema è dibattuto e rigurda in particolare modo l'Italia, abbiamo scelto di svolgere lo studio su tweet in lingua italiana, scaricando dunque solamente i tweet nella lingua selezionata.\\
L'API permette di scaricare tweet con alcune limitazioni relative al numero di giorni e alle richieste di download consecutive. Il periodo d'interesse va dal xx dicembre 2022 al giorno xx gennaio 2023. Il periodo considera dunque anche giorni iniziali in cui alcune voci in merito iniziavano ad emergere e un periodo successivo dopo l'annuncio ufficiale. Come possiamo osservare dalla seguente distribuzione, il giorno con il maggior numero di tweet pubblicati è il giorno xx xxx, giorno in cui...

% img tweet per giorno

Il dataset iniziale, esportato e salvato in csv, si compone dunque di un totale di 12 variabili: ...

\subsection{Data Pre-Processing}
Per ogni documento presente nel corpus, sono state svolte le operazioni di preprocessing necessarie per analizzare il testo. In particolare:
%Tokenization, Punctuation,Rimozione Stop-Words, Rimozione URL


\section{Analysis}
\subsection{Social Network Analysis}
...

\subsection{Social Content Analysis}
...

\section{Visualization}
...

\section{Summary}
...

% risposta domande di ricerca
% per migliorare progetto: più giorni, classificazione neutra sentiment

%%%%%%%%%%%%%%%%%%%%%%%%%%%%%%%%%%%%%%%%%%%%%%%%%%%%%%%%%%%%%%%%%

\nocite{*}
\printbibliography

\end{document}