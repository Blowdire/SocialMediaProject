% Social Media Analytics 2022/23

%%%%%%%%%%%%%%%%%%%%%%%%%%%%%%%%%%%%%%%%%%%%%%%%%%%%%%

% OPTIONAL PACKAGES
\documentclass[12pt,journal,compsoc]{IEEEtran}
\usepackage{amsfonts}
\usepackage{caption}
\usepackage{multirow}
\usepackage[table,xcdraw]{xcolor}
\usepackage{booktabs}
\usepackage{tabularx}
\usepackage[hyphens]{url}
\usepackage{biblatex}
\usepackage{blindtext,graphicx}
\usepackage[absolute]{textpos}
\usepackage[italian]{babel}

\addbibresource{ref.bib} %Import the bibliography file

%%%%%%%%%%%%%%%%%%%%%%%%%%%%%%%%%%%%%%%%%%%%%%%%%%%%%%

\begin{document}

\begin{textblock}{5}(1,0.5)
\noindent\small Social Media Analytics - UniMib 2022/23
\end{textblock}

\title{Smart Working\\
\vspace{2mm}\large{Social Network and Content Analysis}}
\author{Agazzi Ruben 844736\\Cominetti Fabrizio 882737}

\IEEEtitleabstractindextext{%
\begin{abstract}
Negli ultimi anni il tema legato allo \textit{smart working} ha assunto una caratura di primo rilievo nelle discussioni in ambito lavorativo. Dalla necessità nel periodo di pandemia alle richieste di lavoratori che, una volta ristabilità la situazione post-pandemia, non hanno voluto rinunciare a tale possibilità, considerata un vantaggio nel proprio equilibrio tra vita privata e lavoro. Ma il tema è discusso anche per i vantaggi offerti alle aziende, soprattutto in termini di risparmio su strutture ed energie, oppure ancora per questioni legate a traffico e inquinamento. Insomma, lo smart working non può più essere considerato un tema di secondo piano. Per questo motivo, abbiamo voluto analizzare il sentiment e le community degli utenti relativamente al tema, specialmente dopo gli ultimi interventi a livello governativo che hanno parzialmente ridotto la possibilità di usufruire dello smart working nell'ultimo periodo.
\end{abstract}}

\maketitle
\IEEEpeerreviewmaketitle
\IEEEdisplaynontitleabstractindextext

\tableofcontents

%%%%%%%%%%%%%%%%%%%%%%%%%%%%%%%%%%%%%%%%%%%%%%%%%%%%%%

\section{Introduction}
\IEEEPARstart{L}{o} Smart Working, anche conosciuto come lavoro agile, o remote work, indica \textit{una modalità di esecuzione del rapporto di lavoro subordinato caratterizzato dall'assenza di vincoli orari o spaziali e un'organizzazione per fasi, cicli e obiettivi, stabilita mediante accordo tra dipendente e datore di lavoro; una modalità che aiuta il lavoratore a conciliare i tempi di vita e lavoro e, al contempo, favorire la crescita della sua produttività} \cite{MIUR}.\\
A partire da febbraio 2020, a seguito del diffondersi dell’epidemia Covid-19 del Coronavirus, sono stati emanati una serie di provvedimenti per semplificare l’accesso allo Smart Working e diffonderne al massimo l’utilizzo. La pandemia ha infatti costretto numerosi lavoratori a fronteggiare l'emergenza sanitaria tramite lo smart working. Molti di loro però, anche una volta terminata l'emergenza, hanno preferito continuare ad usufruire di tale modalità lavorativa, sottolineandone i numerosi benefici. è facile però trovare anche pareri opposti, tra chi sostiene l'importanza della comunicazione face-to-face e non vorrebbe cedere il passo a nuovi strumenti e tecnologie.\\
In aggiunta a ciò, è anche opportuno precisare che solo alcune tipologie di lavoratori possono svolgere con continuità una modalità di lavoro agile, principalmente i lavoratori d'ufficio, mentre altri basano la propria attività sul contatto e l'esperienza umana e hanno dunque un interesse marginale per l'argomento.\\
Oltre alla già citata pandemia, le direttive annesse per i lavoratori e le varie proroghe allo stato d'emergenza e allo Smart Working in regime semplificato, hanno portato alla ribalta, una volta di più, il fenomeno.\\
Oggi si stimano intorno ai 4 milioni gli smart worker in Italia e quasi il 90\% delle grandi aziende prevede di proseguire con il lavoro agile anche dopo la pandemia \cite{Osservatori}. Lo Smart Working, d'altronde, è un modello organizzativo in grado di portare notevoli vantaggi alle organizzazioni che lo adottano: in termini di produttività, di raggiungimento degli obiettivi, ma anche in termini di welfare e qualità della vita del lavoratore.\\


%Storia recente smart working
La Legge 22 maggio 2017 n. 81 (art. 18-24) \cite{Gazzetta} disciplina il lavoro agile inserendolo in una cornice normativa e fornendo le basi legali per la sua applicazione anche nel settore pubblico.\\


Dopo una serie di proroghe, 

La nostra analisi vuole concentrarsi su questo periodo finale, tra la fine del 2022 e l'inizio del 2023, che ha visto una ulteriore riduzione del diritto allo smart working, ora concesso solamente a...

Addio ai giorni di lavoro da casa per i genitori di figli under 14. Il Governo ha prorogato il diritto allo smart working con un emendamento dedicato in Manovra soltanto per i lavoratori “fragili” del pubblico e del privato. Coloro che hanno figli sotto i 14 anni di età, fino ad oggi compresi nella misura varata dal precedente Esecutivo, dall’1 gennaio dovranno tornare a contrattare individualmente i giorni di lavoro da remoto con la propria azienda. 

La decisione del Governo di escludere dalla misura i genitori di figli under 14 è spiegata dal venire meno delle misure anti-Covid adottate durante la pandemia, sulle quali si basava la necessità di ricorrere al lavoro da remoto per milioni di lavoratori 

%Definire obiettivi analisi, giustificare la scelta del topic e cosa vogliamo fare/indagare



\section{Data Collection}
...

\section{Analysis}
\subsection{Social Network Analysis}
...

\subsection{Social Content Analysis}
...

\section{Visualization}
...

\section{Summary}
...

%%%%%%%%%%%%%%%%%%%%%%%%%%%%%%%%%%%%%%%%%%%%%%%%%%%%%%%%%%%%%%%%%

\nocite{*}
\printbibliography

\end{document}