% Social Media Analytics
% 2022/23

%%%%%%%%%%%%%%%%%%%%%%%%%%%%%%%%%%%%%%%%%%%%%%%%%%%%%%

% OPTIONAL PACKAGES
\documentclass[12pt,journal,compsoc]{IEEEtran}
\usepackage{cite}
\usepackage{amsfonts}
\usepackage{caption}
\usepackage{subcaption}
\usepackage{multirow}
\usepackage[table,xcdraw]{xcolor}
\usepackage{booktabs}
\usepackage{tabularx}
\usepackage{listings}
\usepackage[hyphens]{url}

%%%%%%%%%%%%%%%%%%%%%%%%%%%%%%%%%%%%%%%%%%%%%%%%%%%%%%

\begin{document}
\title{Smart Working\\\normalsize{Social Media Analytics - UniMib 2022/23}}
\author{Agazzi Ruben, Cominetti Fabrizio}

\IEEEtitleabstractindextext{%
\begin{abstract}
...  
\end{abstract}}

\maketitle
\IEEEpeerreviewmaketitle
\IEEEdisplaynontitleabstractindextext

\tableofcontents

%%%%%%%%%%%%%%%%%%%%%%%%%%%%%%%%%%%%%%%%%%%%%%%%%%%%%%

\section{Introduction}
\IEEEPARstart{X}{xxx} Introduction and define the objective of the analysis. Justify the choice of the topic. (What do we want to do?) .

\section{Data Collection}
...

\section{Analysis}
\subsection{Social Network Analysis}
...

\subsection{Social Content Analysis}
...

\section{Visualization}
...

\section{Summary}
...

%%%%%%%%%%%%%%%%%%%%%%%%%%%%%%%%%%%%%%%%%%%%%%%%%%%%%%

\ifCLASSOPTIONcaptionsoff
  \newpage
\fi

\bibliography{ref.bib}
\bibliographystyle{IEEEtran}
\newpage
\onecolumn
\appendices
\section{CovNet Python Code for Learning MNIST}
\label{App:Conv}
\begin{lstlisting}

# 
#  Classifying MNIST with CNNs
# 

import matplotlib.pyplot as plt
import numpy as np

from keras.datasets import mnist
from keras.models import Sequential
from keras.optimizers import SGD, RMSprop
from keras.utils import np_utils

\end{lstlisting}

\end{document}